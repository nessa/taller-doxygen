% ===================================================================
%                   Presentación con Latex Beamer
% ===================================================================
\documentclass[12pt,xcolor=svgnames]{beamer}
%\documentclass[handout,xcolor=svgnames]{beamer} %Version imprimible
% -------------------------------------------------------------------
% Paquetes personalizados
\usepackage{paquetes}
\usepackage{colores}
\usepackage{modo}
\usepackage{licencia}
\usepackage{eurosym}
\usepackage{graphics,epsfig, subfigure}
\usepackage{minted}
\setbeamertemplate{navigation symbols}{} 
\setbeamertemplate{footline}[frame number]

% -------------------------------------------------------------------
% Info
% ====
\title[Doxygen]{Documentación automática\\ con Doxygen}
\author[Noelia Sales]{Noelia Sales Montes}
\institute[DV - UCA]{Diseño de Videojuegos\\
Universidad de Cádiz}
\date{}
% -------------------------------------------------------------------
\logo{\includegraphics[width=1.5cm]{./img/logo}}

% Comienza el documento
\begin{document}
% Tikz -> Imágenes
\tikzstyle{every picture}+=[remember picture]
% Entorno matemático
\everymath{\displaystyle}
% -------------------------------------------------------------------

\begin{frame}
 \titlepage
\end{frame}

\begin{frame}
\frametitle{Índice} 
\transboxin
\tableofcontents
\end{frame}

\section{Introducción}

\begin{frame}{Doxygen}
  Doxygen es...
\end{frame}

\begin{frame}{¿Quién usa Doxygen?}
  \begin{itemize}
  \item Asterisk
  \item Gaim
  \item GNU (Standard C++ Library)
  \item GraphViz
  \item OGRE
  \item Pingus
  \item ScummVM
  \item Synergy
  \item ...\footnote{{\scriptsize \url{http://www.stack.nl/~dimitri/doxygen/projects.html}}}
  \end{itemize}
\end{frame}

\begin{frame}[fragile]{Instalación}
  GNU/Linux - Distribuciones tipo Debian:
  \begin{minted}[fontsize=\footnotesize]{bash}
$ aptitude install doxygen texlive graphviz
$ aptitude install doxygen-gui doxymacs
  \end{minted}    
  \begin{itemize}
  \item $\LaTeX$
  \end{itemize}
\end{frame}


\begin{frame}{Formatos de salida}
  \begin{block}{Directamente soportados}
    \begin{description}
    \item[HTML] \texttt{GENERATE\_HTML = YES}
    \item[$\LaTeX$] \texttt{GENERATE\_LATEX = YES}
    \item[Unix Man] \texttt{GENERATE\_MAN = YES}
    \item[XML] \texttt{GENERATE\_XML = YES}
    \end{description}
  \end{block}
  \begin{block}{Indirectamente soportados}
    \begin{description}
    \item[PDF]
      \begin{itemize}
      \item \texttt{ENABLE\_PDFLATEX = YES}
      \item \texttt{PDF\_HYPERLINKS = YES}
      \end{itemize}
    \end{description}
  \end{block}
\end{frame}


\section{Conclusión}

\begin{frame}{Referencias}
  \scriptsize
  \begin{thebibliography}{6}
  \bibitem{} \url{http://www.stack.nl/~dimitri/doxygen/}
  \end{thebibliography}
\end{frame}

\licencia


\end{document}
