% ===================================================================
%                   Presentación con Latex Beamer
% ===================================================================
\documentclass[12pt,xcolor=svgnames]{beamer}
%\documentclass[handout,xcolor=svgnames]{beamer} %Version imprimible
% -------------------------------------------------------------------
% Paquetes personalizados
\usepackage{paquetes}
\usepackage{colores}
\usepackage{modo}
\usepackage{licencia}
\usepackage{eurosym}
\usepackage{graphics,epsfig, subfigure}
\setbeamertemplate{navigation symbols}{} 
\setbeamertemplate{footline}[frame number]

% -------------------------------------------------------------------
% Info
% ====
\title[Doxygen]{Documentación automática con Doxygen}
\author[Noelia Sales]{Noelia Sales Montes}
\institute[DV - UCA]{Diseño de Videojuegos\\
Universidad de Cádiz}
\date{}
% -------------------------------------------------------------------
\logo{\includegraphics[width=1.5cm]{./img/logo}}

% Comienza el documento
\begin{document}
% Tikz -> Imágenes
\tikzstyle{every picture}+=[remember picture]
% Entorno matemático
\everymath{\displaystyle}
% -------------------------------------------------------------------

\begin{frame}
 \titlepage
\end{frame}

\begin{frame}
\frametitle{Índice} 
\transboxin
\tableofcontents
\end{frame}

\section{Introducción}

\begin{frame}{Doxygen}
  Doxygen es...
\end{frame}


\section{Conclusión}

\begin{frame}{Referencias}
  \begin{thebibliography}{6}
  \bibitem{} \url{http://www.stack.nl/~dimitri/doxygen/}
  \end{thebibliography}
\end{frame}

\licencia

\end{document}
